\chapter*{ABSTRAK} \addcontentsline{toc}{chapter}{ABSTRAK}

\vspace{-2em} % Spasi vertikal untuk judul

\begin{table}[htbp]
\begin{tabular}{l l p{10cm}}
	Nama &:& \penulis\\
	Program Studi &:& \jurusan \\
	Judul &:& \judul
\end{tabular}
\end{table}

Penelitian ini bertujuan untuk merancang dan mengimplementasikan sistem monitoring koefisien restitusi bola berbasis \textit{Internet of Things} (IoT) menggunakan sensor ultrasonik HC-SR04 dan mikrokontroler ESP8266 yang terintegrasi dengan protokol MQTT untuk komunikasi real-time. Sistem dikembangkan untuk mengatasi keterbatasan metode konvensional yang rentan terhadap kesalahan manusia dan tidak memungkinkan monitoring real-time, serta memberikan alternatif yang lebih efisien dibandingkan metode video tracking yang memerlukan analisis pascaproses kompleks. Penelitian melakukan analisis komprehensif terhadap lima jenis material bola berbeda: bola bekel, bola tenis meja, bola tenis lapangan, bola sepak karet, dan bola plastik melalui 100 percobaan (20 percobaan per material). Hasil pengujian menunjukkan sistem mencapai tingkat ketelitian rata-rata 95,84\% dengan eliminasi kesalahan manusia dalam pengukuran manual. Analisis material menghasilkan koefisien restitusi: bola bekel (0,89 ± 0,03, ketelitian 95,84\%), bola tenis meja (0,89 ± 0,04, ketelitian 95,86\%), bola tenis lapangan (0,77 ± 0,05, ketelitian 92,89\%), bola sepak karet (0,78 ± 0,06, ketelitian 91,72\%), dan bola plastik (0,68 ± 0,10, ketelitian 82,45\%). Implementasi algoritma real-time dalam ESP8266 memungkinkan perhitungan koefisien restitusi secara otomatis dengan latensi rata-rata 23 ms, sementara protokol MQTT memberikan stabilitas transmisi data dengan tingkat keberhasilan 98,7\%. Validasi sistem dengan metode referensi menghasilkan korelasi R² = 0,94 dan reproducibility ±2,3\%, memenuhi standar untuk aplikasi pendidikan. Sistem berhasil mengintegrasikan teknologi modern dalam pembelajaran fisika dengan memberikan monitoring real-time, akurasi tinggi, aksesibilitas data, dan visualisasi yang mendukung pembelajaran interaktif. Penelitian ini memberikan kontribusi signifikan dalam modernisasi pendidikan fisika melalui integrasi teknologi IoT, membuka peluang pengembangan sistem pembelajaran yang lebih interaktif, akurat, dan efisien untuk mendukung transformasi digital dalam pendidikan sains dan teknologi.

\noindent\textbf{Kata Kunci:} Koefisien restitusi, \textit{Internet of Things} (IoT), HC-SR04, ESP8266, sensor ultrasonik, fisika, elastisitas, pembelajaran interaktif