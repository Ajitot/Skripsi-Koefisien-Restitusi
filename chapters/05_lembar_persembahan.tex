\chapter*{LEMBAR PERSEMBAHAN}
\addcontentsline{toc}{chapter}{LEMBAR PERSEMBAHAN}

\noindent
\textbf{Dengan penuh rasa syukur, skripsi ini kupersembahkan untuk:} \\
\begin{enumerate}
    \setlength{\leftmargin}{0.5cm}
    \setlength{\itemindent}{-0.5cm}
    \item Kedua orangtuaku tercinta, yang tak pernah lelah memberikan cinta, dukungan, dan doa yang tak terhingga.
    \item Kakak-kakakku tersayang, yang senantiasa memberikan semangat dan menjadi inspirasi dalam setiap langkahku.
    \item Para sahabat setia, yang telah berbagi suka duka dan mewarnai perjalanan hidupku dengan ketulusan.
    \item Para guru dan dosen, yang telah membimbing dengan penuh kesabaran dan mengajarkan begitu banyak ilmu berharga.
    \item Rekan-rekan seperjuangan di Jurusan Fisika, yang telah menemani dan berbagi pengalaman berharga selama masa studi.
    \item Semua pihak yang telah berkontribusi dalam penyelesaian skripsi ini, yang tidak dapat disebutkan satu per satu.
    \item Setiap jiwa yang telah menginspirasi perjalanan hidupku hingga saat ini.
\end{enumerate}

\begin{center}
\textit{"Dan seandainya pohon-pohon di bumi menjadi pena dan laut (menjadi tinta), ditambahkan kepadanya tujuh laut (lagi) sesudah (kering)nya, niscaya tidak akan habis-habisnya (dituliskan) kalimat Allah. Sesungguhnya Allah Maha Perkasa lagi Maha Bijaksana."} \\
(Q.S. Luqman: 27)
\end{center}


\begin{flushright}
    Bandung, \bulan \\
    \vspace{1.5cm}  % Add space between date and signature
    \textit{Penulis}  % Italicize the signature line
\end{flushright}
