\chapter*{ABSTRACT} \addcontentsline{toc}{chapter}{ABSTRACT}

\vspace{-2em} % Spasi vertikal untuk judul

\begin{table}[htbp]
\begin{tabular}{l l p{10cm}}
    Name            & : & \penulis \\
    Department & : & \jurusanInggris     \\
    Title           & : & \judulInggris
\end{tabular}
\end{table}

This research aims to design and implement an IoT-based ball restitution coefficient monitoring system using HC-SR04 ultrasonic sensors and ESP8266 microcontrollers integrated with MQTT protocol for real-time communication. The system was developed to address the limitations of conventional methods that are prone to human error and do not allow real-time monitoring, as well as providing a more efficient alternative compared to video tracking methods that require complex post-processing analysis. The research conducted a comprehensive analysis of five different ball materials: steel ball, ping pong ball, tennis ball, rubber soccer ball, and plastic ball through 100 experiments (20 experiments per material). Test results show the system achieves an average accuracy of 95.84\% with elimination of human error in manual measurements. Material analysis yielded restitution coefficients: steel ball (0.89 ± 0.03, accuracy 95.84\%), ping pong ball (0.89 ± 0.04, accuracy 95.86\%), tennis ball (0.77 ± 0.05, accuracy 92.89\%), rubber soccer ball (0.78 ± 0.06, accuracy 91.72\%), and plastic ball (0.68 ± 0.10, accuracy 82.45\%). Implementation of real-time algorithms in ESP8266 enables automatic calculation of restitution coefficients with an average latency of 23 ms, while MQTT protocol provides data transmission stability with a success rate of 98.7\%. System validation with reference methods resulted in R² = 0.94 correlation and ±2.3\% reproducibility, meeting standards for educational applications. The system successfully integrates modern technology in physics education by providing real-time monitoring, high accuracy, data accessibility, and visualization that supports interactive learning. This research makes a significant contribution to the modernization of physics education through IoT technology integration, opening opportunities for developing more interactive, accurate, and efficient learning systems to support digital transformation in science and technology education.

\textbf{Keywords: Restitution coefficient, Internet of Things (IoT), HC-SR04, ESP8266, ultrasonic sensor, physics, elasticity, interactive learning.}