\chapter{PENUTUP}

\section{Kesimpulan}

Berdasarkan hasil penelitian dan pembahasan, sistem pengukuran koefisien restitusi berbasis \textit{IoT} yang dikembangkan mampu mengatasi keterbatasan metode konvensional. Sistem ini memberikan pemantauan waktu nyata dengan tingkat ketelitian rata-rata 95,84\%, mengurangi kesalahan manusia, serta memudahkan pemahaman hasil penelitian melalui visualisasi data. Pengukuran koefisien restitusi dapat dilakukan secara langsung tanpa analisis \textit{post-processing} rumit, berkat algoritma pada \textit{ESP8266} yang memungkinkan perhitungan otomatis dengan latensi rata-rata 23 ms. Sistem berbasis sensor ultrasonik \textit{HC-SR04} dan \textit{ESP8266} berhasil diimplementasikan untuk pengukuran waktu nyata dan otomatis, didukung protokol \textit{MQTT} yang memberikan stabilitas transmisi data dengan tingkat keberhasilan 98,7\%. Optimasi pada resolusi sensor (±0,5 cm), frekuensi \textit{sampling} (20 Hz), stabilitas komunikasi, dan kalibrasi sistem (faktor koreksi 1,02) meningkatkan performa sistem.

Sistem ini juga berhasil diterapkan dalam pembelajaran fisika, khususnya untuk memahami konsep tumbukan dan elastisitas material, dengan keunggulan pemantauan waktu nyata, akurasi tinggi, akses data mudah, dan visualisasi interaktif. Penelitian ini menganalisis karakteristik koefisien restitusi lima jenis bola: bola bekel (0,89 ± 0,03, ketelitian 95,84\%), tenis meja (0,89 ± 0,04, 95,86\%), tenis lapangan (0,77 ± 0,05, 92,89\%), sepak karet (0,78 ± 0,06, 91,72\%), dan plastik (0,68 ± 0,10, 82,45\%). Bola bekel dan tenis meja memiliki elastisitas tertinggi, diikuti bola sepak karet, tenis lapangan, dan plastik; material elastis menghasilkan pengukuran lebih konsisten dan akurat.

Evaluasi 100 percobaan (20 per jenis bola) menunjukkan konsistensi baik, dengan variabilitas terendah pada bola bekel (±0,03) dan tertinggi pada bola plastik (±0,10), serta \textit{repeatability} baik (koefisien variasi 3,4\%–14,7\%). Validasi sistem menggunakan metode referensi menghasilkan korelasi R² = 0,94, membuktikan akurasi sangat baik, dan \textit{reproducibility} pengukuran (variabilitas ±2,3\%) memenuhi standar aplikasi pendidikan.

\section{Saran}

Beberapa saran untuk pengembangan lebih lanjut antara lain: peningkatan resolusi sensor atau penggunaan beberapa sensor untuk akurasi lebih tinggi, memperluas jenis material bola yang diuji agar data lebih beragam, serta pengembangan algoritma (misal \textit{machine learning}) untuk prediksi koefisien restitusi berdasarkan karakteristik material dan lingkungan. Pengembangan antarmuka \textit{web} atau aplikasi \textit{mobile} terintegrasi dengan \textit{LMS} dapat mendukung pembelajaran jarak jauh dan \textit{hybrid}, serta penyusunan \textit{SOP} untuk implementasi sistem di berbagai institusi pendidikan agar hasil konsisten dan dapat direproduksi.

Penelitian lebih lanjut diperlukan untuk mengetahui pengaruh faktor lingkungan (suhu, kelembapan, tekanan udara) terhadap akurasi sensor ultrasonik, serta integrasi pengukuran parameter fisik lain (massa, diameter, kekerasan bola) untuk analisis lebih komprehensif. Pengembangan sistem berbasis \textit{cloud} dapat dilakukan untuk penyimpanan dan analisis data skala besar, memungkinkan perbandingan hasil antar institusi. Validasi sistem pada skala lebih besar dengan melibatkan banyak institusi pendidikan diperlukan untuk memastikan konsistensi dan reliabilitas. Penyusunan modul pembelajaran terstruktur dan terintegrasi dengan sistem \textit{IoT} ini dapat membantu proses pembelajaran fisika di sekolah maupun perguruan tinggi. Penelitian lanjutan juga dapat menganalisis hubungan sifat fisik material (densitas, \textit{modulus elastisitas}, struktur internal) dengan nilai koefisien restitusi dan ketelitian pengukuran, serta mengembangkan sistem kompensasi otomatis terhadap faktor eksternal (suhu ruangan, arah pantulan, kondisi permukaan lantai) untuk meningkatkan akurasi. Penelitian ini diharapkan berkontribusi pada modernisasi pendidikan fisika melalui integrasi teknologi \textit{IoT} dan menjadi model pengembangan sistem pembelajaran yang lebih interaktif, akurat, dan efisien di masa mendatang.
