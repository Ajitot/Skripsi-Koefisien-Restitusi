\chapter*{KATA PENGANTAR} \addcontentsline{toc}{chapter}{KATA PENGANTAR}


\paragraph{}
Puji syukur kehadirat Allah Subhanallahu Wa Ta'ala atas segala karunianya yang telah memberikan penulis kemudahan dan kelancaran untuk menyelesaikan Skripsi ini. Adapun tujuan penyusunan Skripsi ini adalah sebagai syarat untuk memperoleh gelar sarjana. Tersusunnya Skripsi ini tentu tidak hanya buah kerja keras penulis sendiri, melainkan juga atas bantuan dari berbagai pihak. Untuk itu, penulis mengucapkan terima kasih kepada seluruh pihak yang telah membantu dalam proses penyelesaian laporan ini, terutama :

\begin{enumerate}
\item Puji dan syukur senantiasa penulis panjatkan ke hadirat Allah SWT, Tuhan Yang Maha Esa, atas limpahan rahmat, karunia, dan hidayah-Nya sehingga penulis dapat menyelesaikan tugas akhir ini dengan baik. Shalawat serta salam senantiasa tercurah kepada junjungan kita, Nabi Muhammad SAW, suri teladan sepanjang masa dalam menuntun umat menuju kehidupan yang penuh berkah.

\item Ucapan terima kasih penulis sampaikan kepada Bapak Mada Sanjaya W.S., M.Si., Ph.D. selaku Dosen Pembimbing I dan Bapak Dr. Yudha Satya Perkasa, M.Si. selaku Dosen Pembimbing II, yang telah dengan sabar membimbing, memberikan ilmu, arahan, nasihat, motivasi, serta dukungan yang sangat berarti dalam proses penelitian dan penyusunan tugas akhir ini.

\item Penulis juga mengucapkan terima kasih yang sebesar-besarnya kepada kedua orang tua tercinta, Bapak Andi Permana dan Ibu Iho Hodijah, serta adik penulis, Cepi Perdiana, dan seluruh keluarga besar, atas doa yang tiada henti, kasih sayang yang tulus, serta dukungan baik secara moril maupun materil, yang menjadi sumber kekuatan dalam menyelesaikan karya ini.

\item Rasa terima kasih yang mendalam penulis haturkan kepada almarhum nenek dan kakek, yang semasa hidupnya senantiasa memberikan motivasi dan semangat untuk menjadi pribadi yang sukses. Walaupun mereka telah berpulang ke rahmatullah dan tidak sempat menyaksikan pencapaian ini, doa dan kenangan mereka akan selalu menjadi penyemangat dalam setiap langkah.

\item Penulis juga menyampaikan rasa terima kasih kepada seluruh rekan-rekan mahasiswa Fisika angkatan 2021 di UIN Sunan Gunung Djati Bandung, yang telah berjuang bersama dalam menempuh pendidikan, saling mendukung, dan memberikan semangat sepanjang perjalanan akademik ini.
\end{enumerate}

Penulis menyadari bahwa skripsi ini masih jauh dari kata sempurna. Untuk itu penulis menerima dengan terbuka semua kritik dan saran yang membangun, agar skripsi ini bisa tersusun lebih baik lagi. Penulis berharap, skripsi ini dapat bermanfaat bagi semua pihak terutama untuk penulis dan secara umum bagi pembaca.
 
\vspace{1cm}

\begin{flushright}
    \begin{minipage}{5cm} % Menentukan lebar minipage
        \centering % Menyelaraskan teks di dalam minipage ke tengah
        Bandung, \tahun % Menampilkan tanggal
        \vspace{2cm} % Menambahkan ruang antara tanggal dan tanda tangan
        \\
        Penulis
    \end{minipage}
\end{flushright}
