% Metadata for Thesis
% Author Information
\newcommand{\namapenulis}{Aji Muhamad Pranata}
\newcommand{\nimpenulis}{1217030004}

% Thesis Information
\newcommand{\judul}{Analisis Koefisien Restitusi Bola Menggunakan Sistem Monitoring Real-Time Berbasis ESP8266 dengan Protokol MQTT}
\newcommand{\judulInggris}{Analisis Koefisien Restitusi Bola Menggunakan Sistem Monitoring Real-Time Berbasis ESP8266 dengan Protokol MQTT}

\newcommand{\tahun}{2025}
\newcommand{\bulan}{Mei}

% Department and Faculty Information
\newcommand{\jurusan}{Fisika}
\newcommand{\jurusanInggris}{\textit{Physics}}
\newcommand{\fakultas}{Fakultas Sains dan Teknologi}
\newcommand{\universitas}{UIN Sunan Gunung Djati Bandung}

\newcommand{\penulis}{Aji Muhamad Pranata}
\newcommand{\nimpenulis}{1217030004}

\newcommand{\jeniskaryatulis}{Skripsi} 
% Pilihan: "Laporan KMT" atau "Skripsi"

\newcommand{\jenisSidang}{Sidang Munaqasyah}
% Pilihan:
%1. Sidang KMT
% 2. Sidang Proposal
% 3. Sidang Kolokium
% 4. Sidang Munaqosyah

\newcommand{\tanggalSidang}{20 Juni 2025}

% Faculty Members Information
\newcommand{\nurul}{Dr.\ Moh.Nurul Subhki}
\newcommand{\nipnurul}{1981020120091213003}
\newcommand{\bebeh}{Dr.\ Bebeh Wahid Nuryadin}
\newcommand{\nipbebeh}{19860816201101109}
\newcommand{\khoerunnisa}{Khoerun Nisa Syaja'ah, M.Si}
\newcommand{\nipkhoerunnisa}{199403142022014042}
\newcommand{\imamal}{Dr.\ rer.nat. Imamal Muttaqien}
\newcommand{\nipimamal}{198310062009121009}
\newcommand{\ridwan}{Ridwan Ramdani, M.Si}
\newcommand{\nipridwan}{198904162019031016}
\newcommand{\yudha}{Dr.\ Yudha Satya Perkasa, M.Si.}
\newcommand{\nipyudha}{197911172011011005}
\newcommand{\mada}{Mada Sanjaya W.S., M.Si., Ph.D.}
\newcommand{\nipmada}{198510112009121005}
\newcommand{\hasniah}{Prof.\ Dr.\ Hasniah Aliah, M.Si}
\newcommand{\niphasniah}{197806132005012014}
\newcommand{\fitria}{Fitria Ayu Sulistiani, M.Sc}
\newcommand{\nipfitria}{199504282025052001}

\newcommand{\dekan}{\hasniah}
\newcommand{\nipdekan}{\niphasniah}
\newcommand{\ketuajurusan}{\mada}
\newcommand{\nipketuajurusan}{\nipmada}

% Contoh Penggunaan
\newcommand{\pembimbingsatu}{\mada}
\newcommand{\nippembimbingsatu}{\nipmada}

\newcommand{\pembimbingdua}{\yudha}
\newcommand{\nippembimbingdua}{\nipyudha}

\newcommand{\pengujisatu}{\fitria}
\newcommand{\nippengujisatu}{\nipfitria}

\newcommand{\pengujidua}{\nurul}
\newcommand{\nippengujidua}{\nipnurul}